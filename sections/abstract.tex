\chapter*{}

\section*{Abstrakt}

Tato práce shrnuje současný stav využití metod strojového učení v počítačové bezpečnosti.
Prezentuje způsoby jak může být strojové učení využito jako obraný mechanizmus proti útokům, včetně příkladů systémů, vytvořených předními společnostmi v oboru počítačové bezpečnosti.
Následně je v práci prozkoumáno téma zranitelností a~možností útoků na modely strojového učení samotné, spolu s možnostmi zneužití takovýchto technik ve škodlivém \textit{software} za účelem kradení dat, či skrytí přítomnosti \textit{malware} v napadeném systému.
V~následující kapitole jsou analyzovány techniky klasifikace URL a~je implementováno několik klasifikátorů, které rozhodují o škodlivosti, nebo nezávadnosti předložených URL adres.
V~poslední části je prezentován framework pro vysvětlování závěrů modelů strojového učení.

\vspace{5mm}

\textbf{Klíčová slova} strojové učení, počítačová bezpečnost, URL, klasifikace

\vspace{20mm}

\section*{Abstract}

The thesis summarizes current state of machine learning and cybersecurity.
The ways how machine learning can be used to protect against adversary attacks are presented, with examples of defense systems constructed by IT security companies.
In the same part the topic of machine learning models vulnerabilities and means of exploitation to help malicious software steal data and hide its presence are explored.
In the next chapter, the URL classification techniques are analysed and several classifiers are implemented in order to decide on harmfulness or harmlessness of given URLs.
In the end, the framework explaining machine learning models' conclusions and its possible applications is investigated.

\vspace{5mm}

\textbf{Keywords} machine learning, cybersecurity, URL, classification
