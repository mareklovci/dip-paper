\chapter{Introduction}\label{ch:introduction}

In our highly disruptive society over 1 trillion MB of data is generated daily. \url{https://techjury.net/blog/how-much-data-is-created-every-day/#gref}
All industries are generating data and are in need of using those data to analyze, manage and control various systems.

\section{Ideas}

\subsection{Data Quality Rules}

There are a number of general data quality rules one can deduce from a Feedback-Control Systems view of information systems~\cite{10.1145/269012.269023}.

\begin{enumerate}
    \item Unused data cannot remain correct for very long;
    \item data quality in an information system is a function of its use, not its collection;
    \item data quality will, ultimately, be no better than its most stringent use;
    \item data quality problems tend to become worse as the system ages;
    \item the less likely some data attribute (element) is to change, the more traumatic it will be when if finally does change;
    \item laws of data quality apply qually to data and metadata.
\end{enumerate}

\noindent In principle, we can classify three types of system stability:

\begin{enumerate}
    \item Stable System (Absolute and Conditional stability);
    \item Marginally Stable System;
    \item Unstable System.
\end{enumerate}

\subsection{Analytics Types}

\begin{itemize}
    \item Descriptive – what happened in the past;
    \item Diagnostic – why something happened in the past;
    \item Predictive – what is most likely to happen in the future;
    \item Prescriptive – recommends actions to affect those outcomes.
\end{itemize}

\subsection{Differential Privacy}

%Artificial intelligence is one of the most promising technologies of our times~\cite{ai_citizens}.

%In the chapter~\ref{ch:machine-learning-and-cybersecurity}, I will discuss the ML model security risks and convolution of machine learning, cybersecurity and enterpise solutions.
%Chapter~\ref{ch:malicious-url-detection-and-classification} is going to present available datasets and on page~\pageref{sec:text-feature-extraction} will be demonstrated approach for malicious URL detection.
%The third chapter on page~\pageref{ch:model-analysis} introduces the model explanation framework.
%As a~practical example of use of machine learning for cybersecurity with ability to justify its claims, I am going to make an application in Python programming language using machine learning library \textit{scikit-learn} and LIME (Local Interpretable Model-agnostic Explanations) framework for explaining classification of URL addresses.
