\chapter{Introduction}\label{ch:introduction}

Today, we live in what many call the Information Age, where digital data production is at the heart of all ecosystems.
Businesses of all sizes — large and small — are using some form of data to drive growth.
All industries are in need of using those data to analyze, manage and control various systems.
In our highly disruptive society over 1~trillion (\( 10^{12} \)) MB of data is generated daily~\cite{techjury2021}.
However, one of the downsides of this huge production is poor data quality, causing yearly losses of an estimated \$3.1~trillion in the US alone.
But how do we know, those data are objetively correct?

As Orr (1998) stated,
\blockquote[][]{data quality is the measure of the agreement between the data views presented by an information system and the same data in the real world.
A~system's data quality of 100\% would indicate, for example, that our data views are in perfect agreement with the real world, whereas a data quality rating of 0\% would indicate no agreement at all~\cite{orr1998}.}

Ensuring certain level of data quality is an IT project like any other.
There exists some iniciatives for improving especially Open Data Quality, e.g. 5 Star Data, but none of them is really comprehensive.
In short, objective evaluation of Data Quality is hard.

Main issue with this topic is that DQ is shrouded in misconceptions.
Data Quality is a~business problem, not an IT problem.
But IT enables the business to improve it through tools and processes.
Bad data impacts every system and every person interating with them.
Therefore it should be everybody's responsibility to maintain good standards and practices that in turn will increase confidence in data used for reporting and analytics.

There are two main causes of failure in Data Quality Management implementation.
The first~one is related to missing data quality process, e.g. lack of proactive DQ surveillance~\cite{risto2011}.
The second one is a~lack of data quality measurements~\cite{haug2013}.

% Data Quality Assessment (DQA) and Master Data Management (MDM) are both connected to assuring level of DQ, in this body of work we are going to use them interchangeably.

The cost of bad data is defined as \textit{direct} + \textit{indirect} costs.
Direct costs are, for example, manual and automatic cleaning of master data.
Indirect cost, on the other hand, is financial loss caused by poor-quality master data resulting in
\begin{enumerate*}[label=(\roman*)]
    \item inadequate managerial decision,
    \item process failure and
    \item missed opportunity.
\end{enumerate*}

Even though the goal of DQA is to lower \textit{cost} \& \textit{complexity}, the DQ process can still provide low quality master data.
The usual culprit in such cases is already mentioned
\begin{enumerate*}[label=(\roman*)]
    \item lack of DQ measurements (or faulty definition) and
    \item absence of clear roles in the data life-cycle process~\cite{haug2013}.
\end{enumerate*}

The solution to the problems mentioned above are
\begin{enumerate*}[label=(\roman*)]
    \item a data model definition (metadata) and
    \item proactive data quality surveillance~\cite{risto2011}.
\end{enumerate*}

\blockquote[][]{One certain way to improve the quality of data: improve its use!}~\cite{orr1998}

\section*{Focus of work}

There are many things that must be taken into account when implementing the DQ methodology.
To name a few~\cite{loshin2008}:

\begin{itemize}
    \item stakeholders and participants' involvement,
    \item metadata management,
    \item architecture styles,
    \item functional services,
    \item data modeling,
    \item data consolidation and integration,
    \item management guidance,
    \item master data identification and
    \item master data synchronization.
\end{itemize}

Batini et al. (2009) recognized activities of data quality methodology.
In the most general case, the list is composed of four phases listed below~\cite{batini2009}.
Each of the four sections has additional steps defined, but we will not discuss them explicitly here.
\begin{enumerate}
    \item \textit{State reconstruction}, which is aimed to get information about business processes and services, data collection, quality issues, and corresponding costs.
    \item \textit{Measurement}, where the objective is to measure the quality of data collection along relevant quality dimensions.
    \item \textit{Assessment}, which refers to the event when \textit{measurements} are compared with certain reference values to determine the state of quality and to assess the causes of poor data.
    \item \textit{Improvement} concerns the selection of the steps, strategies, and techniques for reaching new data quality targets.
\end{enumerate}

To limit the scope, in order to achieve a good and reasonable goal within a limited given time and resource, this thesis will focus on these parts:
\begin{enumerate*}[label=(\roman*)]
    \item Measurement of Quality,
    \item Data Quality Score Assesment.
\end{enumerate*}

\subsection*{Measurement of Quality}

Measurement of Quality (MoQ) is part of the \textit{measurement phase}.
The idea is to select the quality dimensions affected by the quality issues identified in the DQ requirements analysis and define corresponding metrics~\cite{loshin2008}.
Measurement can be objective when it is based on quantitative metrics, or subjective, when it is based on qualitative evaluations by data administrators and users~\cite{loshin2008}.

\subsection*{Data Quality Score Assesment}

To achieve the completition ofthe assignment, we will focus on automatic assesment of the final dataset score.
We will define a procedure to objectively evaluate the quality of the given dataset.
This way we will be able to asign a mark to the dataset giving an idea of the current qualitative state.

\section*{Research Goal}

The main research goal of this study is to define and use methodology for data quality assesment.
Therefore to identify, collect, analyze, and evaluate the quality metrics for a~data to allow quantifying and improve their value.
The selected quality metrics should be the ones objectively quantifying data value.

In the applied section of work, we are going to use, test and evaluate data from the (currently) ongoing COVID-19 pandemic.
The reason of choice is general availability, vast collection and opened licence of selected datasets.
