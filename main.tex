\documentclass[a4paper,11pt,openany,oneside]{book}
\usepackage[
a4paper,
left=3.5cm,
right=2.5cm,
top=3cm,
bottom=3.5cm,
bindingoffset=.5cm
]{geometry}

\usepackage[utf8]{inputenc}
\usepackage[T1]{fontenc}
\usepackage{lmodern} % use vector font, not the bitmap one

\usepackage{graphicx}
\usepackage{amsfonts}
\usepackage{amsmath}
\usepackage{mathtools}
\usepackage{amssymb}
\usepackage[inline]{enumitem}
\usepackage{placeins}
\usepackage[group-separator={,}]{siunitx}
\usepackage{caption}
\usepackage{subcaption}
\usepackage{multicol}
\usepackage{textcomp}

\usepackage[backend=biber,style=numeric]{biblatex}
\addbibresource{bibliography.bib}

\usepackage{listings}

% appendices
\usepackage{appendix}

% quotations
\usepackage{csquotes}

% for typesetting urls
\usepackage{hyperref}
\urlstyle{sf}

% tabulky
\usepackage{booktabs}
\usepackage{tabularx,ragged2e,longtable}
\usepackage{multirow}

% tikz graphs and images
\usepackage{tikz}

% for blank page
\usepackage{scrlayer}

\usepackage{color}

% BEGIN LISTINGS (JSON)
% \usepackage{bera}
\usepackage{xcolor}

\colorlet{punct}{red!60!black}
\definecolor{delim}{RGB}{20,105,176}
\colorlet{numb}{magenta!60!black}

\lstdefinelanguage{json}{
    basicstyle=\ttfamily\footnotesize,
    numbers=left,
    numberstyle=\scriptsize,
    stepnumber=1,
    numbersep=8pt,
    showstringspaces=false,
    breaklines=true,
    frame=lines,
    inputencoding=utf8,  % Input encoding
    extendedchars=true,  % Extended ASCII
    literate=        % Support additional characters
     *{0}{{{\color{numb}0}}}{1}
      {1}{{{\color{numb}1}}}{1}
      {2}{{{\color{numb}2}}}{1}
      {3}{{{\color{numb}3}}}{1}
      {4}{{{\color{numb}4}}}{1}
      {5}{{{\color{numb}5}}}{1}
      {6}{{{\color{numb}6}}}{1}
      {7}{{{\color{numb}7}}}{1}
      {8}{{{\color{numb}8}}}{1}
      {9}{{{\color{numb}9}}}{1}
      {:}{{{\color{punct}{:}}}}{1}
      {,}{{{\color{punct}{,}}}}{1}
      {\{}{{{\color{delim}{\{}}}}{1}
      {\}}{{{\color{delim}{\}}}}}{1}
      {[}{{{\color{delim}{[}}}}{1}
      {]}{{{\color{delim}{]}}}}{1},
}
% END LISTINGS (JSON)

% declare argmin and argmac
\DeclareMathOperator*{\argmax}{arg\,max}
\DeclareMathOperator*{\argmin}{arg\,min}

% import acronyms, show glossary in TOC
%\usepackage[acronym,toc]{glossaries}

\setcounter{biburlnumpenalty}{9000} % break on URL numbers
\setcounter{biburllcpenalty}{9000} % break on URL lower case letters
\setcounter{biburlucpenalty}{9000} % break on URL UPPER CASE letters

%\setlength{\parskip}{1em} %paragraph spacing
\renewcommand{\baselinestretch}{1.15} % MS Word 1.15 spacing

%\makeglossaries
%\loadglsentries{sections/glossary}

% blank page declaration
\DeclareNewLayer[
    foreground,
    textarea, % use only the textarea
    contents={
        \parbox[b][\layerheight][c]{\layerwidth}
        {\centering The space above and below the message intentionally is left blank.}
    }
    ]{blankpage.fg}
\DeclarePageStyleByLayers{blank}{blankpage.fg}

% blank page command - usage: \blankpage
\newcommand{\blankpage}{\newpage\null\thispagestyle{blank}\newpage}

% define norm brackets
\newcommand{\norm}[1]{\left\lVert#1\right\rVert}

% signature line
\newcommand{\sign}[2]{%
\begin{tabular}[t]{@{}r@{}}
    \\[-2ex]
    \makebox[#1]{\dotfill}\\
    \strut#2\strut
\end{tabular}%
}

\newcommand{\Date}[1]{%
\begin{tabular}[t]{@{}p{#1}@{}}
    \\[-2ex]
    \strut \dotfill\strut
\end{tabular}%
}

\begin{document}
    \frontmatter

    \linespread{1}

\begin{titlepage}

    \newcommand{\HRule}{\rule{\linewidth}{0.5mm}} % Defines a new command for the horizontal lines, change thickness here

    \center{} % Center everything on the page

    %----------------------------------------------------------------------------------------
    %	HEADING SECTIONS
    %----------------------------------------------------------------------------------------

    \textsc{\LARGE University of West Bohemia}\\[.5cm] % Name of your university/college
    \textsc{\Large Faculty of Applied Sciences}\\[.5cm] % Name of your faculty
    \textsc{\Large Department of Computer Science and Engineering}\\[1.5cm] % Name of your department

    \textsc{\Large master's thesis}\\[0.5cm] % Major heading such as course name
    \textsc{\large KIV/DIP}\\[0.5cm] % Minor heading such as course title

    %----------------------------------------------------------------------------------------
    %	TITLE SECTION
    %----------------------------------------------------------------------------------------

    \HRule{} \\[0.4cm]
    {\huge \bfseries Methodology Design for Dataset Quality Assessment}\\ % Title of your document
    \HRule{} \\[1.5cm]

    %----------------------------------------------------------------------------------------
    %	AUTHOR SECTION
    %----------------------------------------------------------------------------------------

    \begin{minipage}[t]{0.4\textwidth}
        \begin{flushleft}
            \large \emph{Author:}\\
            Bc.\ Marek \textsc{Lovčí}
        \end{flushleft}
    \end{minipage}
    \begin{minipage}[t]{0.4\textwidth}
        \begin{flushright}
            \large \emph{Supervisor:}\\
            Doc.\ Dr.\ Ing.\ Jana \textsc{Klečková}
        \end{flushright}
    \end{minipage}\\[7.5cm] % [3.5cm] with DATE SECTION

    %----------------------------------------------------------------------------------------
    %	DATE SECTION
    %----------------------------------------------------------------------------------------

    % {\large \today}\\[2cm] % Date, change the \today to a set date if you want to be precise
    %{\large \today}\\[2cm]

    %----------------------------------------------------------------------------------------
    %	LOGO SECTION
    %----------------------------------------------------------------------------------------

    \includegraphics[width=80mm,scale=0.5]{imgs/logo.jpg}\\[1cm] % Include a department/university logo - this will require the graphicx package

    %----------------------------------------------------------------------------------------

    \vfill
    % Fill the rest of the page with whitespace

\end{titlepage}


    \chapter*{}

\section*{Návrh metodiky pro vyhodnocení kvality datových sad}

\subsection*{Zadání}

\begin{enumerate}
    \item Seznamte se se současnými metodikami pro vyhodnocení kvality datových sad.
    \item Navrhněte metodiku umožňující univerzální strukturovaný postup pro ohodnocení\linebreak datasetů.
    \item Ověřte možnost automatické klasifikace zvolených datových sad z hlediska kvality.
    \item Proveďte zhodnocení dosažených výsledků.
\end{enumerate}

\vspace{3.5cm}

\section*{Design of a Dataset Quality Assessment Methodology}

\subsection*{Assignment}

\begin{enumerate}
    \item Learn about current methodologies for assessing the quality of datasets.
    \item Create a methodology that allows for a universal structured procedure for dataset evaluation.
    \item Examine the ability to classify selected datasets for quality automatically.
    \item Analyze the results obtained.
\end{enumerate}

    \chapter*{}

\section*{Poděkování}\label{sec:poděkování}

Tímto bych rád poděkoval Doc.\ Dr.\ Ing.\ Janě \textsc{Klečkové} za odborné vedení, za cenné rady a~čas, který strávila čtením a~konzultací této práce.

\vspace{5cm}

\section*{Prohlášení}

Předkládám tímto k posouzení a~obhajobě diplomovou práci zpracovanou na závěr studia na Fakultě aplikovaných věd Západočeské univerzity v Plzni.\newline

Prohlašuji, že jsem diplomovou práci vypracoval samostatně a~výhradně s~použitím odborné literatury a~pramenů, jejichž úplný seznam je její součástí.

\textit{I hereby declare that this master's thesis is completely my own work and that I used only the cited sources.}

\vspace{3.5cm}

\noindent\makebox[\textwidth][c]{%
\begin{minipage}[t]{0.5\textwidth}
    \begin{flushleft}
        Pilsen \Date{1.5in}
    \end{flushleft}
\end{minipage}%
\begin{minipage}[t]{0.5\textwidth}
    \begin{flushright}
        \sign{2in}{Marek \textsc{Lovčí}}
    \end{flushright}
\end{minipage}%
}

    \chapter*{}

\section*{Abstrakt}

Tato práce shrnuje současný stav využití metod strojového učení v počítačové bezpečnosti.
Prezentuje způsoby jak může být strojové učení využito jako obraný mechanizmus proti útokům, včetně příkladů systémů, vytvořených předními společnostmi v oboru počítačové bezpečnosti.
Následně je v práci prozkoumáno téma zranitelností a~možností útoků na modely strojového učení samotné, spolu s možnostmi zneužití takovýchto technik ve škodlivém \textit{software} za účelem kradení dat, či skrytí přítomnosti \textit{malware} v napadeném systému.
V~následující kapitole jsou analyzovány techniky klasifikace URL a~je implementováno několik klasifikátorů, které rozhodují o škodlivosti, nebo nezávadnosti předložených URL adres.
V~poslední části je prezentován framework pro vysvětlování závěrů modelů strojového učení.

\vspace{5mm}

\textbf{Klíčová slova} strojové učení, počítačová bezpečnost, URL, klasifikace

\vspace{20mm}

\section*{Abstract}

The thesis summarizes current state of machine learning and cybersecurity.
The ways how machine learning can be used to protect against adversary attacks are presented, with examples of defense systems constructed by IT security companies.
In the same part the topic of machine learning models vulnerabilities and means of exploitation to help malicious software steal data and hide its presence are explored.
In the next chapter, the URL classification techniques are analysed and several classifiers are implemented in order to decide on harmfulness or harmlessness of given URLs.
In the end, the framework explaining machine learning models' conclusions and its possible applications is investigated.

\vspace{5mm}

\textbf{Keywords} machine learning, cybersecurity, URL, classification


    \tableofcontents
    \listoffigures
    \listoftables

    \mainmatter
    \chapter{Introduction}\label{ch:introduction}

In our highly disruptive society over 1 trillion MB of data is generated daily. \url{https://techjury.net/blog/how-much-data-is-created-every-day/#gref}
All industries are generating data and are in need of using those data to analyze, manage and control various systems.

\section{Ideas}

\subsection{Data Quality Rules}

There are a number of general data quality rules one can deduce from a Feedback-Control Systems view of information systems~\cite{10.1145/269012.269023}.

\begin{enumerate}
    \item Unused data cannot remain correct for very long;
    \item data quality in an information system is a function of its use, not its collection;
    \item data quality will, ultimately, be no better than its most stringent use;
    \item data quality problems tend to become worse as the system ages;
    \item the less likely some data attribute (element) is to change, the more traumatic it will be when if finally does change;
    \item laws of data quality apply qually to data and metadata.
\end{enumerate}

\noindent In principle, we can classify three types of system stability:

\begin{enumerate}
    \item Stable System (Absolute and Conditional stability);
    \item Marginally Stable System;
    \item Unstable System.
\end{enumerate}

\subsection{Analytics Types}

\begin{itemize}
    \item Descriptive – what happened in the past;
    \item Diagnostic – why something happened in the past;
    \item Predictive – what is most likely to happen in the future;
    \item Prescriptive – recommends actions to affect those outcomes.
\end{itemize}

\subsection{Differential Privacy}

\subsection{Quality Classification}

The original idea was to leverage some Machine Learning classification algorithm to automatically classify datasets.
During thesis elaboration the referential materials turned out to be insufficient in providing usefull information on the topic, hence different technique was chosen (composite statistical score-card evaluation).
Shortcoming of white-papers about Machine Learning supported DQ classification probably results from the absence of well-defined general DQA algorithm and output classes.
Complexity of developing all-embracing method for DQA competes with unfolding general artificial intelligence, indeed.

Resulting system should be similar to machine learning ensemble methods (ensemble voting).

%Artificial intelligence is one of the most promising technologies of our times~\cite{ai_citizens}.

%In the chapter~\ref{ch:machine-learning-and-cybersecurity}, I will discuss the ML model security risks and convolution of machine learning, cybersecurity and enterpise solutions.
%Chapter~\ref{ch:malicious-url-detection-and-classification} is going to present available datasets and on page~\pageref{sec:text-feature-extraction} will be demonstrated approach for malicious URL detection.
%The third chapter on page~\pageref{ch:model-analysis} introduces the model explanation framework.
%As a~practical example of use of machine learning for cybersecurity with ability to justify its claims, I am going to make an application in Python programming language using machine learning library \textit{scikit-learn} and LIME (Local Interpretable Model-agnostic Explanations) framework for explaining classification of URL addresses.


    \chapter{Machine Learning and Cybersecurity}\label{ch:machine-learning-and-cybersecurity}

\begin{figure}[htb]
    \centering
    \includegraphics[width=0.9\textwidth]{figures/dq-simple.png}
    \caption{}
    \label{fig:dq-simple}
\end{figure}
\FloatBarrier

\begin{figure}[htb]
    \centering
    \includegraphics[width=0.9\textwidth]{figures/dq-system.png}
    \caption{}
    \label{fig:dq-system}
\end{figure}
\FloatBarrier

\section{Data Quality Attributes}

Eppler (2006) presented list of seventy of the most used data and information quality criteria explicitly defined in the literature.
They provide criterial basis for most of the DQ frameworks.
The list is shown in the figure~\ref{fig:dq-criteria}.

\begin{figure}[htb]
    \begin{multicols}{3}
        \begin{enumerate}
            \item Comprehensiveness
            \item Accuracy
            \item Clarity
            \item Applicability
            \item Conciseness
            \item Consistency
            \item Correctness
            \item Currency
            \item Convenience
            \item Timeliness
            \item Traceability
            \item Interactivity
            \item Accessibility
            \item Security
            \item Maintainability
            \item Speed
            \item Objectivity
            \item Attributability
            \item Value-added
            \item Reputation (source)
            \item Ease-of-use
            \item Precision
            \item Comprehensibility
            \item Trustworthiness\newline (source)
            \item Reliability
            \item Price 
            \item Verifiability
            \item Testability
            \item Provability
            \item Performance
            \item Ethics
            \item Privacy
            \item Helpfulness
            \item Neutrality
            \item Ease of Manipulation
            \item Validity
            \item Relevance
            \item Coherence
            \item Interpretability
            \item Completeness
            \item Learnability
            \item Exclusivity
            \item Right Amount
            \item Existence of meta information
            \item Appropriateness\newline of meta information
            \item Target group orientation
            \item Reduction of complexity
            \item Response time
            \item Believability
            \item Availability
            \item Consistent Representation
            \item Ability to represent null values
            \item Semantic Consistency
            \item Concise Representation
            \item Obtainability
            \item Stimulating
            \item Attribute granularity
            \item Flexibility
            \item Reflexivity
            \item Robustness
            \item Equivalence of redundant or distributed data
            \item Concurrency of redundant or distributed data
            \item Nonduplication
            \item Essentialness
            \item Rightness
            \item Usability
            \item Cost
            \item Ordering
            \item Browsing
            \item Error rate
        \end{enumerate}
    \end{multicols}

    \centering
    \caption{Data \& Information Quality Criteria~\cite{eppler2006}}
    \label{fig:dq-criteria}
\end{figure}
\FloatBarrier

Uniqueness
Accuracy
Consistency
Completeness
Timeliness
Currency
Format Compliance
Referential Integrity

% \section{AI as a~tool for security breaches mitigation}\label{sec:ai-as-a-tool-for-security-breaches-mitigation}

%In the section~\ref{subsec:evasion} the principle of \textbf{model verifiability} was already mentioned.

\subsection{Parallel hybrid systems}\label{subsec:parallel-hybrid-systems}

In the paper \say{Is machine learning cybersecurity's silver bullet?} ESET's experts sort training data into three groups - malicious, clean and potentially unwanted.
They recommend not to use algorithm own output data as inputs, because any further errors are reinforced and multiplied, as the same incorrect result enters a~loop and creates more false positives or misses of malicious items.
In the next chapter ESET points out how crucial it is to achieve an equilibrium of sufficient protection from malicious items and false positives minimised to a~manageable level.
Finally, in the chapter \textit{Machine learning by ESET - The road to augur} authors let us take a~look under the hood of their ML engine called Augur.
For malicious file detection they are using two branches:
\begin{enumerate*}[label=(\roman*)]
    \item sandbox analysis followed by advanced memory analysis and behavioural features extraction (these features are later used to train ML models),
    \item ML-based branch.
\end{enumerate*}
ML-based branch consists of two methodologies:
\begin{enumerate*}[label=(\roman*)]
    \item neural networks, specifically deep learning and long short-term memory (LSTM),

    \item consolidated output of six classification algorithms.
\end{enumerate*}
While consolidating output of those six classification algorithms, two modes (setups) are used.
The first one is used for security critical environments, making algorithm more likely to mark file as malicious if most of the previous algorithms vote it as such.
The other setup is more conservative - labelling a~sample clean if at least one algorithm comes to such conclusion.

\subsection{Serial hybrid systems}\label{subsec:serial-hybrid-systems}

\begin{enumerate}[label=(\roman*)]
    \item exposure prevention (network filtering),
    \item pre-execution detection based on machine learning,
    \item runtime control proactively looking out for suspicious behavior of devices in the network (behavioral analysis based on ML),
    \item automated response such as \textit{automatic rollback} to help restore systems to their pre-attack state, system disinfection techniques or Incident of Compromise (IoC) scanning~\cite{whitepaper:kaspersky_next_generation}.
\end{enumerate}

Features are extracted from items in hard regions, to undergo ML classifiaction.
Various types of models are pre-trained with human annotated data.
Which model is selected for classification of items in region depends on several factors - extractable features, type of objects, etc.

Second topic relate to data integrity.
There is serious concern that attackers can inject data while a~model is in the training stage to alter the inference capability or add disturbance into the input samples to change model's interpretation and distort result.

Other option is to append additional component to capture and therefore filter out malicious input sample before it gets into inference stage - \textbf{adversarial sample detection}.
Simple deterministic detector could be a~deterministic comparer having some type of \textit{distance} as a~criterion.
Detectors vary greatly - forming a~group of independent models worth exploring in other papers.
% possibility to write more about detectors - the detection model may extract related information at each layer of the original model to perform detection based on the extracted information

A deformed input samples does not effect normal classification function of a~model.
\textbf{Input reconstruction} works by deforming input samples to defend against evansion attack by adding noise, de-noising, or using an automatic encoder (a type of artificial neural network)~\cite{huawei_security}.

Last but not least method is \textbf{model verification}.
In general, verification is a~discipline of software engineering with goal to assure that software fully satisfies all the expected requirements.

\textbf{Regression analysis} methods such as linear and ordinary least squares regression are ideal to detect noise and abnormalities in the data sets.
Thanks to relative directness presents those methods easy way to fight back data poisoning attack.

\textbf{Ensemble analysis} points out that usage of multiple sub-models - each one of them trained with different training data set - reduces probability of system being affected by poisoning attacks greatly.

\begin{enumerate}
    \item \textbf{Explainable data}
    As Huawei in its paper mentions, if several representative characteristics can be found at data sets and those features are carefully selected, then a~some models can be meaningfully interpretted~\cite{huawei_security}.
    Of course, data set are not usually simple enough to make such analysis.
    Moreover, AI model can grow in complexity and even with understandable data and features in the beggining, there is no guarantee that result is intereprettable in the end.

    \item \textbf{Explainable model}
    Some of the ML models (either for classification or regression) are interprettable naturaly.
    Their typical properties are \textit{linearity}, \textit{monotonicity} and \textit{interaction features} - possibility to manually add non-linearity into the model.
\end{enumerate}

\subsection{Data security}\label{subsec:data-security}

In addition to model and architecture security mechanism, we have to consider security of data sets.
Data often contains personal information of users - so called \textit{sensitive attributes}.
Such information can be in a~form of personal identifiers or quasi-identifiers.
\textit{Personal identifier} is an unique information that identifies a~user - a~birth number, bank account number and other types of personal IDs.
\textit{Quasi-identifiers} are characteristics which needs to be used in combination with others to identify an entity - examples are gender, postal code, age or nationality.

To prevent data stealing and following re-identification of users, several models exists to protect personal information of individuals in dataset.
Those privacy models are \textbf{optimal k-anonymity}, \textbf{l-diversity}, \textbf{t-closeness} and \textbf{differential privacy}.
Three general types of attack to datasets exists:
\begin{enumerate*}[label=(\roman*)]
    \item \label{itm:reident} re-identifying an individual,
    \item \label{itm:query} query whether an individual is a~member of a~dataset,
    \item \label{itm:linking} linking an individual to a~sensitive attribute.
\end{enumerate*}

\textbf{Optimal k-anonymity} protects against both cases~\ref{itm:reident} and~\ref{itm:query} by transforming quasi-identifiers so that at least \( k - 1 \) members of set are indistinguishable from each other - group based anonymization.
Identifiers are transformed by suppression (needless attributes are replaced with \textit{dummy} values) and generalization (individual values of attributes are replaced with a~broader category - e.g.\ specific age can be replaced by a~range).
As \( k \) increases risk of data exploit reduces, on the other hand data quality decreases - we are talking about a~\textit{privacy-utility} tradeoff.
Moreover, the \( k \) is limit - in order for this method to work if \( k \) is set to \( k \triangleq 10 \) then any group must contain at least \( 10 \) individuals.
The first drawback of k-anonymity is vulnerability to \textit{homogeneity attack} which works on premise of data having sensitive value identical within a~set of \( k \) records - it is enough to find the group of records, the individual belongs to, if all of them have the same sensitive value.
Second drawback is the possibility of \textit{background knowledge attack} where attacker identifies associations among one or more quasi-identifiers and reduces the set of possible values for the sensitive attribute.

Both \textbf{l-diversity} and \textbf{t-closeness} are group based anonymization techniques building on a~concept of \textbf{optimal k-anonymity}.
In addition to \textbf{optimal k-anonymity}, \textbf{T-closeness} transforms quasi-identifiers such that each group is within a~distance \( t \) of the distribution of sensitive values for the entire dataset~\cite{web:privacy-models}.
The distance is measured as the cumulative absolute difference of the distributions, as \( t \) decreases both risk of sensitive attribute disclosure and data quality decreases.
Suppose that the sensitive attribute is salary.
Each group's frequency distribution of salary will be within a~distance \( t \) from the salary frequency distribution for the entire dataset~\cite{web:privacy-models}.

\textbf{Differential privacy} and its variants (epsilon, epsilon-delta) are statistical techniques aiming to protect data against \textit{differentiated attack}.
The model guarantees that even if someone has complete information about 99 of 100 people in a~data set, they still cannot deduce the sensitive information about the final person~\cite{web:differential-privacy}.
The mechanism works by adding random noise to the aggregate data, leaving only a~trend without possibility to figure out exact values in data (e.g.\ information that \( n\% \) of users prefer some product over another).

    \chapter{Methodology}\label{ch:methodology}

Data Quality methodology consists of two main parts, the model and the metamodel.
The model defines activities, their description, goals and sequence of order needed for extraction of Data Quality metrics.
The metamodel defines components of DQ process, maps the relationship among the components and activities respectively.

\section{Data Quality and Security}

In addition to model and architecture security mechanism, we have to consider security of data sets.
Data often contains personal information of users - so called \textit{sensitive attributes}.
Such information can be in a~form of personal identifiers or quasi-identifiers.
\textit{Personal identifier} is an unique information that identifies a~user - a~birth number, bank account number and other types of personal IDs.
\textit{Quasi-identifiers} are characteristics which needs to be used in combination with others to identify an entity - examples are gender, postal code, age or nationality.

To prevent data stealing and following re-identification of users, several models exists to protect personal information of individuals in dataset.
Those privacy models are \textbf{optimal k-anonymity}, \textbf{l-diversity}, \textbf{t-closeness} and \textbf{differential privacy}.
Three general types of attack to datasets exists:
\begin{enumerate*}[label=(\roman*)]
    \item \label{itm:reident} re-identifying an individual,
    \item \label{itm:query} query whether an individual is a~member of a~dataset,
    \item \label{itm:linking} linking an individual to a~sensitive attribute.
\end{enumerate*}

\textbf{Optimal k-anonymity} protects against both cases~\ref{itm:reident} and~\ref{itm:query} by transforming quasi-identifiers so that at least \( k - 1 \) members of set are indistinguishable from each other - group based anonymization.
Identifiers are transformed by suppression (needless attributes are replaced with \textit{dummy} values) and generalization (individual values of attributes are replaced with a~broader category - e.g.\ specific age can be replaced by a~range).
As \( k \) increases risk of data exploit reduces, on the other hand data quality decreases - we are talking about a~\textit{privacy-utility} tradeoff.
Moreover, the \( k \) is limit - in order for this method to work if \( k \) is set to \( k \triangleq 10 \) then any group must contain at least \( 10 \) individuals.
The first drawback of k-anonymity is vulnerability to \textit{homogeneity attack} which works on premise of data having sensitive value identical within a~set of \( k \) records - it is enough to find the group of records, the individual belongs to, if all of them have the same sensitive value.
Second drawback is the possibility of \textit{background knowledge attack} where attacker identifies associations among one or more quasi-identifiers and reduces the set of possible values for the sensitive attribute.

Both \textbf{l-diversity} and \textbf{t-closeness} are group based anonymization techniques building on a~concept of \textbf{optimal k-anonymity}.
In addition to \textbf{optimal k-anonymity}, \textbf{T-closeness} transforms quasi-identifiers such that each group is within a~distance \( t \) of the distribution of sensitive values for the entire dataset~\cite{web:privacy-models}.
The distance is measured as the cumulative absolute difference of the distributions, as \( t \) decreases both risk of sensitive attribute disclosure and data quality decreases.
Suppose that the sensitive attribute is salary.
Each group's frequency distribution of salary will be within a~distance \( t \) from the salary frequency distribution for the entire dataset~\cite{web:privacy-models}.

\textbf{Differential privacy} and its variants (epsilon, epsilon-delta) are statistical techniques aiming to protect data against \textit{differentiated attack}.
The model guarantees that even if someone has complete information about 99 of 100 people in a~data set, they still cannot deduce the sensitive information about the final person~\cite{web:differential-privacy}.
The mechanism works by adding random noise to the aggregate data, leaving only a~trend without possibility to figure out exact values in data (e.g.\ information that \( n\% \) of users prefer some product over another).

\section{Business Problems and Data Defects}

% Není možné mluvit o kvalitě dat a metodikách pro její zajištění, bez zmínky o konkrétních typech problémů, které se v rámci tématu vyskytují.
It is not possible to talk about data quality and methodologies for ensuring it, without mentioning the specific types of problems that occur within the topic.
There is quite a few common defects in the field of data engineering and data science.

\subsection*{Missing Data}

This is data that does not reach the destination data store.
This problem usually occurs when handling the data needed to clean up in the source database; by operating with invalid or incorrect lookup table in the transformation logic; or by invalid table joins.

\paragraph*{Example} We transform data from Task Management Solution.
Lookup table should contain a field value of \enquote{Minor} which maps to \enquote{Low}.
However, source data field contains \enquote{Mino} - missing the \textit{r} and fails the lookup, resulting in the target data field containing null.
If this occurs on a key field, a possible join would be missed and the entire row could fall out.

\subsection*{Truncation of Data}

Many data is being lost by truncation of the data fields.
This happens when there are invalid field lengths on target database or by transformation logic not taking into account field lengths from the source.

\paragraph*{Example} We transform financial data with complete exchange-traded fund (ETF) names.
Source field value \enquote{iShares Global High Yield Corp Bond UCITS ETF} is being truncated to \textit{varchar(32)}.
since the source data field did not have the correct length to capture the entire field, only \enquote{iShares Global High Yield Corp B} is stored.

\subsection*{Data Type Mismatch}

Data types not setup correctly on target database cause serious problems.
This usually happens when using ETL pipeline with an automatic or semi-automatic column type recognition.
% Data inženýr spoléhá na bezchybné rozpoznání datového typu a nezkontroluje správnost výstupních tabulek.
The data engineer relies on error-free data type recognition and does not check the accuracy of the output tables.

\paragraph*{Example} Source data field was required to be a \textit{varchar}, however, when initially configured, was setup as a \textit{date}.

\subsection*{Null Translation}

In the source dataset, \textit{null} values are not being transformed to correct target values.
Development team did not include the \textit{null} translation in the ETL process.

\paragraph*{Example} A \enquote{Null} source data field was supposed to be transformed to \enquote{None} in the target data field.
However, the logic was not implemented, resulting in the target data field containing \enquote{null} values\footnotemark.

\footnotetext{
    None is a concept that describes the absence of anything at all (nothingness), while Null means \textit{unknown} (we do not know if there is a value or not).
}

\subsection*{Wrong Translation}

Wrong translations happen when a source data field for \textit{null} was supposed to be transformed to \enquote{None} in the target data field, but was not transformed correctly.
The logic was not implemented, resulting in the target data field containing \textit{null} values.
Wrong translation is the exact opposite to \textit{Null Translation}.

\paragraph*{Example} Target field should only be populated when the source field contains certain values, otherwise should be set to null.
Let's look at a very basic example.
During analytical processing of medical data (e.g., list of patients with oncological finding), we need to set target field to \textit{true} if the one or multiple source values indicate certain treatment.
However, the target field is populated (either with blank charater or other values) although source values do not correspond to the required logic.

\subsection*{Misplaced Data}

If the source data fields are not being transformed to the correct target data fields, we call the issue \enquote{Misplaced Data}.
One of the possible causes is that development team inadvertently mapped the source data field to the wrong target data field.

\paragraph*{Example} A source data field was supposed to be transformed to target data field \enquote{Last\_Update}.
However, the development team inadvertently mapped the source data field to \enquote{Date\_Created}.

\subsection*{Extra Records}

Records which should be excluded in the ETL are included in the ETL.
This happens when developers do not include filter in their code.

\paragraph*{Example} If a record has the deleted field populated, the record and any data related to that record should not be in any ETL.

\subsection*{Not Enough Records}

Records which should be in the ETL are not included in the ETL.
Development team had a filter in their code which should not have been there.

\paragraph*{Example} If a record was in a certain state, it should be sent through ETL pipeline over to the data warehouse.

\subsection*{Transformation Logic Errors}

Testing sometimes can lead to finding \enquote{holes} in the transformation logic or realizing the logic is unclear.

Sometimes, the processes are just way too complicated and development team does not take into account special cases.
Most cases fall into a certain branch of logic for a transformation, but a small subset of cases (sometimes with unusual data) may not fall into any branches.
How the analytics and developers handles these cases could be different (and may both end up being wrong) and the logic is changed to accommodate the cases.
The next reason why this happens is that analytic and developer have different interpretation of transformation logic, which results in different values.
This leads to the logic being re-written to become clearer.

\paragraph*{Example} International cities that contain special language specific characters might need to be dealt with in the ETL code (e.g., Århus).

\subsection*{Simple and small Errors}

Capitalization, spacing and other small errors cause problems with data.
Data inconsistencies are easy to fix, but happen often.
The only real solution is to always double check data and ETL procedure.

\subsection*{Sequence Generator}

Ensuring that the sequence number of reports are in the correct order is very important when processing follow up reports or answering to an audit.
If the sequence generator is not configured correctly, procedure results in records with a duplicate sequence number.

\paragraph*{Example} Duplicate records in the sales report were doubling up several sales transactions which skewed the report significantly.

\subsection*{Undocumented Requirements}

During ETL development, sometimes certain requirements are found, that are \enquote{understood} but are not actually documented anywhere.
This causes issues when members of the development team do not understand or misunderstood the undocumented requirements.

\paragraph*{Example} ETL pipeline contains a restriction in the \enquote{where} clause, limiting how certain reports are brought over.
Moreover, there were used mappings that were understood to be necessary, but were not actually in the requirements.
Occasionally, it turns out that the understood requirements are not what the business wanted.

\subsection*{Duplicate Records}

Duplicate records are two or more records that contain the same data.
This issue happens when development team does not add the appropriate code to filter out duplicate records or there is some unexpected error in data generators.

\paragraph*{Example} Duplicate records in the sales report were doubling up several sales transactions which skewed the report significantly.

\subsection*{Numeric Field Precision}

Numbers that are not formatted to the correct decimal point or not rounded per specifications cause precision problems.
This has several causes, development team rounded the numbers to the wrong decimal point, used wrong rounding type or used wrong data type which lead to faulty rounding.

\paragraph*{Example} The sales data did not contain the correct precision and all sales were being rounded to the whole dollar.

\subsection*{Rejected Rows}

Data rows that get rejected by ETL process due to data issues.
Development team did not take into account data conditions that break the ETL for a particular row.

\paragraph*{Example} Missing data rows on the sales table caused major issues with the end of year sales report.

% a~\cite{web:common-defects}

\section{Model}

The methodology has several important components that need to be identified or developed.
The metamodel that covers the required components is as depicted in figure~\ref{fig:methodology-metamodel}.
The activities within the process model have a goal to develop those components.

Overall, the methodology consists of two main processes.
The first one is \textbf{Specification Process}.
The goal of this processs is to identify and define context specific ways to measure data quality.
The second one is an \textbf{Execution Process}.
Its main goal is to \textit{collect} and \textit{verify} data with output from \textit{Specification Process} taken into account.

\begin{figure}[htb]
    \centering
    \includegraphics[width=0.7\textwidth]{figures/dq-methodology.png}
    \caption{Methodology Metamodel}
    \label{fig:methodology-metamodel}
\end{figure}
\FloatBarrier

\subsection{Specification Process}

TODO

\subsubsection{Identification}

This activity focuses on identification of systems, processes and business schemes generatig data.
By identifying weak points and bottlenecks in those processes, we can find causes of poor data.
Also, we need to identify the subprocesses or activities that are mostly affected by the product data quality.

\subsubsection{Metrics Specification}

The goal of this activity is to identify the process metrics or KPIs.
Measuring data quality is all about understanding what data quality attributes are and choosing the correct data quality metrics.
Data Quality Attributes were discussed in back in chapter~\ref{ch:literature-review} section~\ref{sec:data-quality-attributes} and will be further discussed in chapter~\ref{ch:quality-classification-system}.

\subsubsection{Proof of Concept Verification}

The last part of current process is verification and Proof of Concept.
This activity has to ensure that selected metrics are meaningful enough, capturing the actual condition of data.

\subsection{Execution Process}

TODO

\subsubsection{Collection}

Data collection is a systematic process of gathering observations or measurements.
Data collector can be either \textit{Information System}, computer program or a human.
Before the beginning of collecting data, we need to consider:

\begin{itemize}
    \item the type of data we will collect;
    \item the methods and procedures we will use to collect, store and process data.
\end{itemize}

\subsubsection{Verification}

In our general case, verification is based on actual reliability of data, computed using DQ metrics.
In other scenarios, the verification could be based on data redundancies, therefore based on the comparison of the collected data from two or more different collectors.
If all data match, the data will be considered as valid.
If not, the data remains invalid until a further collector validates it.

Artificial Intelligence and Machine Learning could be used to further ease and optimize data verification.
Especially when processing image data and data with a high level of abstraction.

\subsubsection{Contract}

TODO

\subsection{Supporting Techniques}

TODO

\subsubsection{Proof of Constancy}

Proof of Constant Data, alias Proof of Constancy, is a way to assure a constant accuracy of data.
Data have to be regularly updated to keep the accuracy rate high.
Data accuracy rate will decrease progressively based on a specific time frame basis (e.g., X\% per month).
This percentage is different depending on the type of data.
Data sets more sensitive to changes may see this rate decrease by 5\% to 10\% per month or day depending on the circumstances.
On the other hand, established, well-known sets, will see their rate decrease by 0.1\% per month or even year.
A scale of discount rates will have to be established based on the areas of interest and actual items collected.

\subsubsection{Proof of Trust}

Proof of Trust is an instrument for data collector evaluation.
The collector or generator will get \enquote*{quality score} for his/her or its collection actions.
The more collectors initiate, update and verify data correctly, the higher their \enquote*{quality score} will be.
A higher quality score leads to a higher level of \enquote*{trust}.
On the other hand, incorrect collection leads to a retroactive decrease of the collector's quality score.

\subsection{Roles}

TODO

\section{Use Cases}

In this part, we will present several use cases, to illustrate versatile use of the presented framework.

\subsection{Enterprise Information System}

Enterprises suffer from poor data quality.
We propose, following the methodology, to introduce a central register of data sources.
This central register should be supported by a set of services and a central data repository.

After a thorough analysis of data requirements and their quality, a defined set of metrics and key performance indicators parameterizes the verification chain of activities.
If the predefined quality limit is not met, the data will either be rejected or saved with an error flag.
% Jestliže data splňují požadovanou úrověň chybovosti, projdou kontraktačním procesem a jsou požadována za referenční až do doby, kdy je jejich poslední verze je nařazeným procesem kvalitativně degradována a označena za nedůvěryhodnou.
If the data meets the required level of error, they go through the contracting process and are considered as a reference until their latest version is qualitatively degraded by the ordered process (e.g., Proof of Constancy) and marked as untrusted.

% Penalizací za špatnou kvalitu by bylo automatické hlášení vyššímu managementu společnosti.
A penalty for poor quality would be automatic reporting to the company's senior management.
% Management by následně mohl uvalit na osoby zodpovědné za konkrétní datové sady a datové toky sankce ve formě snížení nebo zrušení osobních odměn.
Management could then impose sanctions on those responsible for specific datasets and data flows in the form of reductions or cancellations of personal rewards.

\subsection{IoT Cluster}

Based on the domain and usage of the IoT devices, the data repository could be either centralized (e.g., nuclear power plant cluster of secondary senzors) or decentralized (e.g., community weather stations).

The verification algorithm would - in this case - consist from two general authorities.
The first authority being \textit{k} nearest neighbours of the same sensors (or IoT devices in general), and the second one being the set of domain rules.
Nearest neighbors provide redundancy by which data can be verified.
A data samotná musí samozřejmě splňovat kriteriální omezení daná doménou využití.

% Špatnou kvalita by vedla ke snížení významnosti senzoru v klusteru, případně jeho dočasnému nebo úplnému vyřazení z provozu.
Poor quality would reduce the importance of the sensor in the cluster, or its temporary or complete decommissioning.
% Tento systém by vytořil i velice účinnou obranou bariéru proti útokům.
This system would also create a very effective defense barrier against attacks, especially against data poisoning.

Data poisoning is a class of attacks on machine learning algorithm where an adversary alters a fraction of the training data in order to impair the intended function of the system.
Objective can be to degrade the overall accuracy of the trained classifier, escaping security detection or to favor one product over the another.
Machine Learning systems are usually retrained after deployment to adapt to changes in input distribution, so data poisoning represents serious danger.

Qualitative degradation of data by Proof of Constancy would not be the so important, because we expect very high update frequency.
However, lower update frequancy of IoT device would suggest an error within a system, which could serve as a warning to network operators about a faulty device.
Data from defective equipment should also not be taken into account in many cases.

\subsection{Open Data Library}

The last Use Case shows usage of completely decentralized solution.
The system would allow those who collect and generate data to be rewarded and data would be accessible for use within a decentralized marketplace.
This decentralized network would democratize access to data while rewarding those who generate it.

Data collection would be done through an application (system) used by a community of collectors who are rewarded for their actions.
This reward is calculated according to a~\enquote*{collection value} price.

The collection value would be calculated using an algorithm that takes into account several criteria, such as:
\begin{itemize}
    \item demand and rarity,
    \item online availability and accessibility,
    \item data licensing market value.
\end{itemize}

Each collector receives a quality score to maintain a high level of reliability
The verified data is then made accessible (through contracting) on the decentralized marketplace and regularly updated to keep it accurate.

Depending on the data, a blockchain could be used as a storage.
Using the blockchain makes the data unalterable, guaranteeing the transparency and traceability of its validation process (collection, verification, update).
The blockchain (tamper-proof, immutable and decentralized) ensures the integrity and verification of the data available on the marketplace.
This brings confidence and security to the data acquirers that exploit it. 
Using \textbf{Smart Contract} technology also guarantees the rewards of the collectors.

Due to the variety of open data, automation of the verification process is practically impossible.
The data must be verified manually.
Thus, a collector has two functions:
\begin{itemize}
    \item initiate the data collection (input and update data),
    \item verify the data collected (check a collected data not yet verified).
\end{itemize}

The mechanism ensures that reward for data collection is divided between collector and verifier.
For example, the collector, who initiated the data, would obtain 60-80\% of the reward.
The verifier would obtain remaining 20-40\%.

\subsubsection{Smart Contracts}

TODO

    \chapter{Quality Classification System}\label{ch:quality-classification-system}

The original idea was to leverage some Machine Learning classification algorithm to automatically classify datasets.
During thesis elaboration the referential materials turned out to be insufficient in providing usefull information on the topic, hence different technique was chosen (composite statistical score-card evaluation).
Shortcoming of white-papers about Machine Learning supported DQ classification probably results from the absence of well-defined general DQA algorithm and output classes.
Complexity of developing all-embracing method for DQA competes with unfolding general artificial intelligence, indeed.

Resulting system should be similar to machine learning ensemble methods (ensemble voting).

\section{Data Quality Criteria}

Tabular vs relational data
Tabular data structures: Cross-Sectional, Time-Series, Pooled Cross-Sections, Panel (Longitudal)
Metrics usability with raw (non-aggregated) and aggregated data.


In order to provide 

\subsection{Accuracy}

TODO

\subsection{Completeness}

Blake and Mangiameli (2011) defined completeness as follows.
On the level of data values, a~data value is incomplete (i.e., the metric value is zero) if and only if it is \enquote*{NULL}, otherwise it is complete (i.e., the metric value is one).
A tuple in a relation is defined as complete if and only if all data values are complete (i.e., none of its data values is \enquote*{NULL}).
For a relation \( R \), let \( T_R \) be the number of tuples in \( R \) which have at least one \enquote*{NULL}-value and let \( N_R \) be the total number of tuples in \( R \). Then, the completeness \( C \) of \( R \) is defined as follows~\cite{blake2011}.

\begin{equation}
    C = 1 - \frac{T_R}{N_R} = \frac{N_R - T_R}{N_R}
\end{equation}

\subsection{Consistency}

There are several forms of data consistency.
The \textbf{first form} is actual wide or narrow distribution of data.
In this way, consistency of data can be viewed as \textit{stability}, \textit{uniformity} or \textit{constancy}.
Typical measures include statistics such as the \textit{range} (i.e., the largest value minus the smallest value among a distribution of data), the \textit{variance} (i.e., the sum of the squared deviations of each value in a distribution from the mean value in a distribution divided by the number of values in a distribution) and the \textit{standard deviation} (i.e., the square root of the variance).

\begin{figure}[htb]
    \centering
    
    \begin{equation*}
        \sigma = \sqrt{\frac{1}{N}\sum_{i=1}^{N} (x_{i} - \mu)^2}
    \end{equation*}

    \caption{Population Standard Deviation formula}
    \label{form:population-standard-dev}
\end{figure}
\FloatBarrier

\begin{figure}[htb]
    \centering
    
    \begin{equation*}
        s = \sqrt{\frac{1}{N - 1}\sum_{i=1}^{N} (x_{i} - \bar{x})^2}
    \end{equation*}

    \caption{Sample Standard Deviation formula}
    \label{form:sample-standard-dev}
\end{figure}
\FloatBarrier

If one is evaluating the consistency of data drawn in a sample from a population, the \textit{standard error of the mean} (i.e., the standard deviation of the sampled population divided by the square root of the sample size) is often examined.
Finally, the constancy of data produced by instruments and tests is typically measured by estimating the reliability of~obtained scores.
Reliability estimates include test-retest coefficients, split-half measures and Kuder-Richardson Formula \textnumero 20 indexes~\cite{quora:consistency2017}.
For Time Series data, stationary analysis can be done.
If the data is non-stationary then it is likely to have some degree of inconsistency.

\begin{figure}[htb]
    \centering
    
    \begin{equation*}
        \sigma_{\bar{x}} = \frac{\sigma}{\sqrt{n}}
    \end{equation*}

    \caption{Standard Error of the Mean formula}
    \label{form:sem}
\end{figure}
\FloatBarrier

Then, there is \textbf{second form} of data consistency; whether data are uniformly defined throughout the dataset, that is, across variables and over time.
For example, suppose we want to use the data to estimate real estate sales per year to see how that number has changed over time.
In this case, we have to make sure the estimates of real estate sales are uniformly defined over time.
Specifically, does the data series always either include apartments or exclude apartments from the counts?
Does it always either include houses or exclude houses from the counts?
If the data sometimes include apartments, but not always, or if the data sometimes include houses, but not always, then the data are inconsistent.

The \textbf{third form} of consistency tightly coupled with relational databases and their referential integrity.
A relational database is said to be ACID (vs non-relational BASE), meaning
\begin{enumerate*}[label=(\roman*)]
    \item atomicity,
    \item consistency,
    \item isolation and
    \item durability.
\end{enumerate*}
The term onsistency there refers to the requirement that any given database transaction must affect data only in allowed ways, therefore data must be valid according to all defined rules, including constraints, cascades, triggers, and any combination thereof.

Inconsistencies in data can be due to changes over time and/or across variables for example, in
\begin{enumerate*}[label=(\roman*)]
    \item vintages or time periods,
    \item units,
    \item levels of accuracy,
    \item levels of completeness,
    \item inclusions and exclusions.
\end{enumerate*}
Those inconsistencies occur most often when merging or aggregating datasets, therefore the user has to make sure data are consistently defined throughout.

\subsection{Timeliness}

Timeliness is another one of the major dimensions in the field of data quality.
Timely dataset is product of function of the forecast update frequency (a dataset released annualy will be updated only once a year)~\cite{atz2014tau}.

\begin{equation*}
    T = I \frac{f_U}{today - last update}
\end{equation*}

\begin{equation*}
    \tau = \frac{1}{N} \sum_{i = 1}^N I \frac{f_{U_i} \lambda + \delta}{today - {last update}_i}
\end{equation*}

\begin{table}[htbp]
    \centering

    \begin{tabular}{@{}ll@{}}
        \toprule
        \( \tau \)  & Data Timeliness   \\ \midrule
        0.9-1       & exemplar          \\
        0.7-0.9     & standard          \\
        0.5-0.7     & ok                \\
        0.25-0.5    & poor              \\
        0-0.25      & obsolete          \\
        \bottomrule
    \end{tabular}

    \caption{Proposed benchmarks for different levels of \( \tau \)~\cite{atz2014tau}}
    \label{table:timeliness-benchmarks}
\end{table}
\FloatBarrier

    \chapter{Case Study}\label{ch:case-study}

    \chapter{Conclusion and future work}\label{ch:conclusion-and-future-work}

\section{Future work}\label{sec:future-work}

In the paper I opened many topics, unable to explore them in depth.
There are many possible ways, the paper could be augmented.

\subsection{Data Quality Methodologies with Security Constraints}

\section{Conclusion}\label{sec:conclusion}


    \begin{appendices}
        \chapter{Data Quality Attributes}\label{ch:data-quality-attributes}

Eppler (2006) presented list of seventy of the most used data and information quality criteria explicitly defined in the literature.
They provide the basis for most of the DQ frameworks.

\begin{figure}[htb]
    \footnotesize

    \begin{multicols}{3}
        \begin{enumerate}
            \item Comprehensiveness
            \item Accuracy
            \item Clarity
            \item Applicability
            \item Conciseness
            \item Consistency
            \item Correctness
            \item Currency
            \item Convenience
            \item Timeliness
            \item Traceability
            \item Interactivity
            \item Accessibility
            \item Security
            \item Maintainability
            \item Speed
            \item Objectivity
            \item Attributability
            \item Value-added
            \item Reputation (source)
            \item Ease-of-use
            \item Precision
            \item Comprehensibility
            \item Trustworthiness\newline (source)
            \item Reliability
            \item Price 
            \item Verifiability
            \item Testability
            \item Provability
            \item Performance
            \item Ethics
            \item Privacy
            \item Helpfulness
            \item Neutrality
            \item Ease of Manipulation
            \item Validity
            \item Relevance
            \item Coherence
            \item Interpretability
            \item Completeness
            \item Learnability
            \item Exclusivity
            \item Right Amount
            \item Existence of meta information
            \item Appropriateness\newline of meta information
            \item Target group orientation
            \item Reduction of complexity
            \item Response time
            \item Believability
            \item Availability
            \item Consistent Representation
            \item Ability to represent null values
            \item Semantic Consistency
            \item Concise Representation
            \item Obtainability
            \item Stimulating
            \item Attribute granularity
            \item Flexibility
            \item Reflexivity
            \item Robustness
            \item Equivalence of redundant or distributed data
            \item Concurrency of redundant or distributed data
            \item Nonduplication
            \item Essentialness
            \item Rightness
            \item Usability
            \item Cost
            \item Ordering
            \item Browsing
            \item Error rate
        \end{enumerate}
    \end{multicols}

    \centering
    \caption{Data \& Information Quality Criteria~\cite{eppler2006}}
    \label{fig:dq-criteria}
\end{figure}
\FloatBarrier
    
        \chapter{Dataset Collection}\label{ch:dataset-collection}

\section{Epidemiological Characteristics}

\begin{enumerate}
    \item Základní přehled
    \item Přehled osob s prokázanou nákazou dle hlášení krajských hygienických stanic (v2)
    \item Celkový (kumulativní) počet osob s prokázanou nákazou dle krajských hygienických stanic včetně laboratoří (v2)
    \item Přehled vyléčených dle hlášení krajských hygienických stanic
    \item Přehled úmrtí dle hlášení krajských hygienických stanic
    \item Přehled hospitalizací
    \item Celkový (kumulativní) počet osob s prokázanou nákazou dle krajských hygienických stanic včetně laboratoří, počet vyléčených, počet úmrtí a provedených testů (v2)
    \item Přehled epidemiologické situace dle hlášení krajských hygienických stanic podle okresu
    \item Přehled epidemiologické situace dle hlášení krajských hygienických stanic podle ORP
    \item Epidemiologická charakteristika obcí
    \item Epidemiologická charakteristika městských částí hlavního města Prahy
\end{enumerate}

\section{Testing}

\begin{enumerate}
    \item Celkový (kumulativní) počet provedených testů (v2)
    \item Přehled provedených testů podle typu a indikace
    \item Celkový (kumulativní) počet provedených testů podle krajů a okresů ČR
    \item Odběrová místa v ČR
\end{enumerate}

\section{Vaccination}

\begin{enumerate}
    \item Přehled vykázaných očkování podle krajů ČR
    \item Přehled vykázaných očkování podle očkovacích míst ČR
    \item Očkovací místa v ČR
    \item Přehled spotřeby podle očkovacích míst ČR
    \item Přehled distribuce očkovacích látek v ČR
    \item Přehled registrací podle očkovacích míst ČR
    \item Přehled rezervací podle očkovacích míst ČR
    \item Přehled vykázaných očkování podle profesí
\end{enumerate}

\section{Other}

\begin{enumerate}
    \item Přehled distribuce ochranného materiálu dle krajů ČR (v2)
\end{enumerate}

        \chapter{Metadata File Structure}\label{ch:metadata-file-structure}

\begin{figure}[htb]
    \centering
    \begin{lstlisting}[language=json,firstnumber=1]
{
    "@context": [
        "http://www.w3.org/ns/csvw",
        {
            "@language": "cs"
        }
    ],
    "url": "zakladni-prehled.csv",
    "dc:title": "COVID-19: Zakladni prehled",
    "dc:description": "...",
    "dc:source": "Krajske hygienicke stanice v CR",
    "dcat:keyword": ["COVID-19", "widget", "aktualni situace"],
    "dc:publisher": {
        "schema:name": "UZIS CR",
        "schema:url": {
            "@id": "https://www.uzis.cz/"
        }
    },
    "dc:license": {
        "@id": "https://data.gov.cz/podminky-uziti/volny-pristup/"
    },
    "dc:modified": {
        "@value": "2021-04-18",
        "@type": "xsd:date"
    },
    "tableSchema": {
        "columns": [
            {
                "name": "datum",
                "titles": "datum",
                "datatype": "date",
                "dc:description": "Datum vytvoreni aktualizace."
            },...
        ]
    }
}
    \end{lstlisting}

    \caption{Metadata file structure}
    \label{ls:metadata}
\end{figure}
\FloatBarrier

        % \definecolor{codegreen}{rgb}{0,0.6,0}
\definecolor{codegray}{rgb}{0.5,0.5,0.5}
\definecolor{codepurple}{rgb}{0.58,0,0.82}

\lstdefinestyle{mystyle}{
    commentstyle=\color{codegreen},
    keywordstyle=\color{magenta},
    numberstyle=\tiny\color{codegray},
    stringstyle=\color{codepurple},
    basicstyle=\footnotesize,
    breakatwhitespace=false,
    breaklines=true,
    captionpos=b,
    keepspaces=true,
    numbers=left,
    numbersep=5pt,
    showspaces=false,
    showstringspaces=false,
    showtabs=false,
    tabsize=2
}

\lstset{style=mystyle}

\chapter{URL tokenizer}\label{ch:url-tokenizer}

Implementation of algorithm infering the location of spaces in a~string as presented in the original \textit{StackOverflow} question~\cite{stackoverflow:tokenizer}.
The code is written at the Python programming language.

\begin{lstlisting}[language=Python]
    from math import log

    words = open("words-by-frequency.txt").read().split()
    wordcost = dict((k, log((i+1)*log(len(words)))) for i,k in enumerate(words))
    maxword = max(len(x) for x in words)


    def infer_spaces(s):

        def best_match(i):
            candidates = enumerate(reversed(cost[max(0, i-maxword):i]))
            return min((c + wordcost.get(s[i-k-1:i], 9e999), k+1) for k,c in candidates)

        # Build the cost array.
        cost = [0]
        for i in range(1,len(s)+1):
            c,k = best_match(i)
            cost.append(c)

        # Backtrack to recover the minimal-cost string.
        out = []
        i = len(s)
        while i>0:
            c,k = best_match(i)
            assert c == cost[i]
            out.append(s[i-k:i])
            i -= k

        return " ".join(reversed(out))
\end{lstlisting}

        % \chapter{Sources}\label{ch:sources}

\section{Source Code}\label{sec:source-code}

Source code used in the thesis is available at \url{https://github.com/mareklovci/dip-code}.

\section{Source \LaTeX}\label{sec:source-latex}

Source code for thesis text is available at \url{https://github.com/mareklovci/dip-paper}

    \end{appendices}

    %\newpage
    %\printglossary[type=\acronymtype]
    %\printglossary

    \newpage
    \printbibliography

\end{document}
