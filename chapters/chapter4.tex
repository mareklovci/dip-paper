\chapter{Case Study}\label{ch:case-study}

In December 2019, a virus known as COVID-19 was first identified in Wuhan, China~\cite{seznamKorona2021}.
Three months later, on March 1, 2020, the first three cases of the disease were confirmed in the Czech Republic~\cite{seznamKorona2021}.
The disease has shown and continues to show the shortcomings of social and political environment worldwide.
But the disease, although very serious, has given us many opportunities.
One such opportunity is open datasets made available by state institutions.

In the following part of the work we will try to analyze the state of datasets provided by the institutions of the Czech Republic.
We apply the metrics defined in Chapter 3 to the data in order to objectively measure their quality and comment on the results.
Available datasets as of April 1, 2020 from the URL address \url{https://onemocneni-aktualne.mzcr.cz/} are listed in Appendix~\ref{ch:dataset-collection}.

The data can be downloaded via the REST API in JSON, in addition, the data can be downloaded in CSV together with metadata also in JSON format.
The format of the JSON data file can be seen in Figure~\ref{ls:data}.
All data files contain one main object, with three keys (modified, source and data).
The \enquote{modified} key contains the date of the dataset update in ISO 8601 format with the time offset from UTC (Coordinated Universal Time).
The \enquote{source} key contains the URL of the dataset, especially the protocol and domain name.
The last key, the data, contains an array of json objects with a structure given by the metadata.
This is an array of objects even if the array contains only one object.

\begin{figure}[htb]
    \centering
    
    \begin{lstlisting}[language=json,firstnumber=1]
{
    "modified": "2021-04-18T12:28:42+02:00",
    "source": "https:\/\/onemocneni-aktualne.mzcr.cz\/",
    "data": [
        {
            ...
        }
    ]
}
    \end{lstlisting}

    \caption{Data file structure}
    \label{ls:data}
\end{figure}
\FloatBarrier

The structure of the metadata file is shown in Figure~\ref{ls:metadata}.

\begin{figure}[htb]
    \centering
    
    \begin{lstlisting}[language=json,firstnumber=1]
{
    "@context": [
        "http://www.w3.org/ns/csvw",
        {
            "@language": "cs"
        }
    ],
    "url": "zakladni-prehled.csv",
    "dc:title": "COVID-19: Zakladni prehled",
    "dc:description": "...",
    "dc:source": "Krajske hygienicke stanice v CR",
    "dcat:keyword": [
        "COVID-19",
        "widget",
        "aktualni situace"
    ],
    "dc:publisher": {
        "schema:name": "UZIS CR",
        "schema:url": {
            "@id": "https://www.uzis.cz/"
        }
    },
    "dc:license": {
        "@id": "https://data.gov.cz/podminky-uziti/volny-pristup/"
    },
    "dc:modified": {
        "@value": "2021-04-18",
        "@type": "xsd:date"
    },
    "tableSchema": {
        "columns": [
            {
                "name": "datum",
                "titles": "datum",
                "datatype": "date",
                "dc:description": "Datum vytvoreni aktualizace."
            },
            ...
        ]
    }
}
    \end{lstlisting}

    \caption{Metadata file structure}
    \label{ls:metadata}
\end{figure}
\FloatBarrier
