\chapter{Quality Classification System}\label{ch:quality-classification-system}

The original idea was to leverage some Machine Learning classification algorithm to automatically classify datasets.
During thesis elaboration the referential materials turned out to be insufficient in providing usefull information on the topic, hence different technique was chosen (composite statistical score-card evaluation).
Shortcoming of white-papers about Machine Learning supported DQ classification probably results from the absence of well-defined general DQA algorithm and output classes.
Complexity of developing all-embracing method for DQA competes with unfolding general artificial intelligence, indeed.

Resulting system should be similar to machine learning ensemble methods (ensemble voting).

\section{Data Quality Criteria}

Tabular vs relational data
Tabular data structures: Cross-Sectional, Time-Series, Pooled Cross-Sections, Panel (Longitudal)
Metrics usability with raw (non-aggregated) and aggregated data.


In order to provide 

\subsection{Accuracy}

TODO

\subsection{Completeness}

Blake and Mangiameli (2011) defined completeness as follows.
On the level of data values, a~data value is incomplete (i.e., the metric value is zero) if and only if it is \enquote*{NULL}, otherwise it is complete (i.e., the metric value is one).
A tuple in a relation is defined as complete if all data values are complete (i.e., none of its data values is \enquote*{NULL}).
For a relation \( R \), let \( T_R \) be the number of tuples in \( R \) which have at least one \enquote*{NULL}-value and let \( N_R \) be the total number of tuples in \( R \). Then, the completeness \( C \) of \( R \) is defined as follows~\cite{blake2011}.

\begin{equation}
    C = 1 - \frac{T_R}{N_R} = \frac{N_R - T_R}{N_R}
\end{equation}

This definition of \textit{completeness} meets the requirements for metrics according to Heinrich et al. (2018).
The metric values are within the bounded interval \( \left[0; 1\right] \) for all aggregation levels.
The minimum value represents perfectly poor data quality and vice versa.
To archieve full score, no tuple must not contain a null value, as well as relations must not contain any tuple with data values which equal \enquote*{NULL}.

The metric is reliable because all configuration parameters of the metric can be determined by a database query.
Due to the existence of a mathematical formula, the metric is objective and because the metric quantifies a dimension at all quality levels according to the corresponding definition, the determination of the metric value is also valid.
The metric formula is applicable to single data values as well as to sets of data values.

\subsection{Consistency}

There are several forms of data consistency.
The \textbf{first form} is actual wide or narrow distribution of data.
In this way, consistency of data can be viewed as \textit{stability}, \textit{uniformity} or \textit{constancy}.
Typical measures include statistics such as the \textit{range} (i.e., the largest value minus the smallest value among a distribution of data), the \textit{variance} (i.e., the sum of the squared deviations of each value in a distribution from the mean value in a distribution divided by the number of values in a distribution) and the \textit{standard deviation} (i.e., the square root of the variance).

\begin{figure}[htb]
    \centering
    
    \begin{equation*}
        \sigma = \sqrt{\frac{1}{N}\sum_{i=1}^{N} (x_{i} - \mu)^2}
    \end{equation*}

    \caption{Population Standard Deviation formula}
    \label{form:population-standard-dev}
\end{figure}
\FloatBarrier

\begin{figure}[htb]
    \centering
    
    \begin{equation*}
        s = \sqrt{\frac{1}{N - 1}\sum_{i=1}^{N} (x_{i} - \bar{x})^2}
    \end{equation*}

    \caption{Sample Standard Deviation formula}
    \label{form:sample-standard-dev}
\end{figure}
\FloatBarrier

If one is evaluating the consistency of data drawn in a sample from a population, the \textit{standard error of the mean} (i.e., the standard deviation of the sampled population divided by the square root of the sample size) is often examined.
Finally, the constancy of data produced by instruments and tests is typically measured by estimating the reliability of~obtained scores.
Reliability estimates include test-retest coefficients, split-half measures and Kuder-Richardson Formula \textnumero 20 indexes~\cite{quora:consistency2017}.
For Time Series data, stationary analysis can be done.
If the data is non-stationary then it is likely to have some degree of inconsistency.

\begin{figure}[htb]
    \centering
    
    \begin{equation*}
        \sigma_{\bar{x}} = \frac{\sigma}{\sqrt{n}}
    \end{equation*}

    \caption{Standard Error of the Mean formula}
    \label{form:sem}
\end{figure}
\FloatBarrier

Then, there is \textbf{second form} of data consistency; whether data are uniformly defined throughout the dataset, that is, across variables and over time.
For example, suppose we want to use the data to estimate real estate sales per year to see how that number has changed over time.
In this case, we have to make sure the estimates of real estate sales are uniformly defined over time.
Specifically, does the data series always either include apartments or exclude apartments from the counts?
Does it always either include houses or exclude houses from the counts?
If the data sometimes include apartments, but not always, or if the data sometimes include houses, but not always, then the data are inconsistent.

The \textbf{third form} of consistency tightly coupled with relational databases and their referential integrity.
A relational database is said to be ACID (vs non-relational BASE), meaning
\begin{enumerate*}[label=(\roman*)]
    \item atomicity,
    \item consistency,
    \item isolation and
    \item durability.
\end{enumerate*}
The term onsistency there refers to the requirement that any given database transaction must affect data only in allowed ways, therefore data must be valid according to all defined rules, including constraints, cascades, triggers, and any combination thereof.

Inconsistencies in data can be due to changes over time and/or across variables for example, in
\begin{enumerate*}[label=(\roman*)]
    \item vintages or time periods,
    \item units,
    \item levels of accuracy,
    \item levels of completeness,
    \item inclusions and exclusions.
\end{enumerate*}
Those inconsistencies occur most often when merging or aggregating datasets, therefore the user has to make sure data are consistently defined throughout.

\subsection{Timeliness}

Timeliness is another one of the major dimensions in the field of data quality.
Obsolete data suppress innovation, therefore businesses and startups want to trust the data publisher that the data will remain available and relevant, especially when using open data or reference data from central registers.
A measure of timeliness has to focus on the update cycle.
Automation must be a key part of this process, leading to the efficiency in publishing and processing of data.
Meeting all these points is a necessary, but not sufficient condition to create a sustainable data ecosystem.

Atz (2014) proposed an unique metric for measuring the timeliness of data.
The research defines timely dataset as a product of function of the forecast update frequency (a dataset released annualy will be updated only once a year)~\cite{atz2014tau}.
The concept of timeliness \( T \) can be expressed by the equation~\ref{eq:timeliness}.

\begin{equation}\label{eq:timeliness}
    T = I \frac{f_U}{today - last update}
\end{equation}

In the equation~\ref{eq:timeliness}, the \( I \) is an indicator function causing \textit{Heaviside} step function effect returning 1 when the ratio is greater than one and 0 otherwise.
For example, a dataset with a daily cycle and a last major update last month would result in a 0.
On the other hand, a dataset with monthly cycle and an update in last two weeks, would yield 1.
In the equation \( f_U \) represents \textit{update frequency}; the terms \textit{today} and \textit{last update} are timepoints corresponding to the names.

% Důvod přítomnosti indikátorové funkce je ten, že nemáme nástroj, kterým bychom zhodnotili způsob stárnutí datasetu.
The reason for the presence of the indicator function is that we do not have a tool to evaluate the aging of the dataset.
% Data mohou zastarávat lineárně a spojitě, ale také nelineárně a nespojitě.
Data can become obsolete linearly and continuously, but also non-linearly and discontinuously.
% Tato funkční závislost je nám skryta, data považujeme tedy buď za aktuální nebo zastaralá.
This functional dependence is hidden from us, so we consider the data to be either current or obsolete.

% \[ T = \begin{cases} 0 & x\leq 0 \\ \frac{100-x}{100} & 0\leq x\leq 100 \\ 0 & 100\leq x \end{cases} \]

Atz (2014) introduced a metric for measuring data catalogue timeliness, \( \tau \).
In general, minor changes such as typo correction should not be considered as an update.

\begin{equation*}
    \tau = \frac{1}{N} \sum_{i = 1}^N I \left( \frac{f_{U_i} \cdot \lambda + \delta}{today - {last update}_i} \right)
\end{equation*}

The \( \tau \) of a data catalogue is the average across datasets, indicated by the subscript \textit{i}~\cite{atz2014tau}.
The number of datasets in the catalog is denoted by the \textit{N}.

Two parameters in a linear form have been introduced to the core of the expression.
The lambda (\( \lambda \)) is degree of freedom relative to the update frequency; the days we allow the update of the data catalog.
For example, considering 5\% of the time reserve (e.g., due to ETL delays), the annual renewal dataset is going have a buffer of 0.6 months and for a monthly dataset it implies 1.5 days in tolerance.
The delta (\( \delta \)) is a fixed number of days applicable to all datasets, for example one day for processing~\cite{atz2014tau}.

\begin{table}[htbp]
    \centering

    \begin{tabular}{@{}ll@{}}
        \toprule
        \( \tau \)  & Data Timeliness   \\ \midrule
        0.9-1       & exemplar          \\
        0.7-0.9     & standard          \\
        0.5-0.7     & ok                \\
        0.25-0.5    & poor              \\
        0-0.25      & obsolete          \\
        \bottomrule
    \end{tabular}

    \caption{Proposed benchmarks for different levels of \( \tau \)~\cite{atz2014tau}}
    \label{table:timeliness-benchmarks}
\end{table}
\FloatBarrier
